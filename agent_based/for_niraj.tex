
% Default to the notebook output style

    


% Inherit from the specified cell style.




    
\documentclass[11pt]{article}

    
    
    \usepackage[T1]{fontenc}
    % Nicer default font (+ math font) than Computer Modern for most use cases
    \usepackage{mathpazo}

    % Basic figure setup, for now with no caption control since it's done
    % automatically by Pandoc (which extracts ![](path) syntax from Markdown).
    \usepackage{graphicx}
    % We will generate all images so they have a width \maxwidth. This means
    % that they will get their normal width if they fit onto the page, but
    % are scaled down if they would overflow the margins.
    \makeatletter
    \def\maxwidth{\ifdim\Gin@nat@width>\linewidth\linewidth
    \else\Gin@nat@width\fi}
    \makeatother
    \let\Oldincludegraphics\includegraphics
    % Set max figure width to be 80% of text width, for now hardcoded.
    \renewcommand{\includegraphics}[1]{\Oldincludegraphics[width=.8\maxwidth]{#1}}
    % Ensure that by default, figures have no caption (until we provide a
    % proper Figure object with a Caption API and a way to capture that
    % in the conversion process - todo).
    \usepackage{caption}
    \DeclareCaptionLabelFormat{nolabel}{}
    \captionsetup{labelformat=nolabel}

    \usepackage{adjustbox} % Used to constrain images to a maximum size 
    \usepackage{xcolor} % Allow colors to be defined
    \usepackage{enumerate} % Needed for markdown enumerations to work
    \usepackage{geometry} % Used to adjust the document margins
    \usepackage{amsmath} % Equations
    \usepackage{amssymb} % Equations
    \usepackage{textcomp} % defines textquotesingle
    % Hack from http://tex.stackexchange.com/a/47451/13684:
    \AtBeginDocument{%
        \def\PYZsq{\textquotesingle}% Upright quotes in Pygmentized code
    }
    \usepackage{upquote} % Upright quotes for verbatim code
    \usepackage{eurosym} % defines \euro
    \usepackage[mathletters]{ucs} % Extended unicode (utf-8) support
    \usepackage[utf8x]{inputenc} % Allow utf-8 characters in the tex document
    \usepackage{fancyvrb} % verbatim replacement that allows latex
    \usepackage{grffile} % extends the file name processing of package graphics 
                         % to support a larger range 
    % The hyperref package gives us a pdf with properly built
    % internal navigation ('pdf bookmarks' for the table of contents,
    % internal cross-reference links, web links for URLs, etc.)
    \usepackage{hyperref}
    \usepackage{longtable} % longtable support required by pandoc >1.10
    \usepackage{booktabs}  % table support for pandoc > 1.12.2
    \usepackage[inline]{enumitem} % IRkernel/repr support (it uses the enumerate* environment)
    \usepackage[normalem]{ulem} % ulem is needed to support strikethroughs (\sout)
                                % normalem makes italics be italics, not underlines
    

    
    
    % Colors for the hyperref package
    \definecolor{urlcolor}{rgb}{0,.145,.698}
    \definecolor{linkcolor}{rgb}{.71,0.21,0.01}
    \definecolor{citecolor}{rgb}{.12,.54,.11}

    % ANSI colors
    \definecolor{ansi-black}{HTML}{3E424D}
    \definecolor{ansi-black-intense}{HTML}{282C36}
    \definecolor{ansi-red}{HTML}{E75C58}
    \definecolor{ansi-red-intense}{HTML}{B22B31}
    \definecolor{ansi-green}{HTML}{00A250}
    \definecolor{ansi-green-intense}{HTML}{007427}
    \definecolor{ansi-yellow}{HTML}{DDB62B}
    \definecolor{ansi-yellow-intense}{HTML}{B27D12}
    \definecolor{ansi-blue}{HTML}{208FFB}
    \definecolor{ansi-blue-intense}{HTML}{0065CA}
    \definecolor{ansi-magenta}{HTML}{D160C4}
    \definecolor{ansi-magenta-intense}{HTML}{A03196}
    \definecolor{ansi-cyan}{HTML}{60C6C8}
    \definecolor{ansi-cyan-intense}{HTML}{258F8F}
    \definecolor{ansi-white}{HTML}{C5C1B4}
    \definecolor{ansi-white-intense}{HTML}{A1A6B2}

    % commands and environments needed by pandoc snippets
    % extracted from the output of `pandoc -s`
    \providecommand{\tightlist}{%
      \setlength{\itemsep}{0pt}\setlength{\parskip}{0pt}}
    \DefineVerbatimEnvironment{Highlighting}{Verbatim}{commandchars=\\\{\}}
    % Add ',fontsize=\small' for more characters per line
    \newenvironment{Shaded}{}{}
    \newcommand{\KeywordTok}[1]{\textcolor[rgb]{0.00,0.44,0.13}{\textbf{{#1}}}}
    \newcommand{\DataTypeTok}[1]{\textcolor[rgb]{0.56,0.13,0.00}{{#1}}}
    \newcommand{\DecValTok}[1]{\textcolor[rgb]{0.25,0.63,0.44}{{#1}}}
    \newcommand{\BaseNTok}[1]{\textcolor[rgb]{0.25,0.63,0.44}{{#1}}}
    \newcommand{\FloatTok}[1]{\textcolor[rgb]{0.25,0.63,0.44}{{#1}}}
    \newcommand{\CharTok}[1]{\textcolor[rgb]{0.25,0.44,0.63}{{#1}}}
    \newcommand{\StringTok}[1]{\textcolor[rgb]{0.25,0.44,0.63}{{#1}}}
    \newcommand{\CommentTok}[1]{\textcolor[rgb]{0.38,0.63,0.69}{\textit{{#1}}}}
    \newcommand{\OtherTok}[1]{\textcolor[rgb]{0.00,0.44,0.13}{{#1}}}
    \newcommand{\AlertTok}[1]{\textcolor[rgb]{1.00,0.00,0.00}{\textbf{{#1}}}}
    \newcommand{\FunctionTok}[1]{\textcolor[rgb]{0.02,0.16,0.49}{{#1}}}
    \newcommand{\RegionMarkerTok}[1]{{#1}}
    \newcommand{\ErrorTok}[1]{\textcolor[rgb]{1.00,0.00,0.00}{\textbf{{#1}}}}
    \newcommand{\NormalTok}[1]{{#1}}
    
    % Additional commands for more recent versions of Pandoc
    \newcommand{\ConstantTok}[1]{\textcolor[rgb]{0.53,0.00,0.00}{{#1}}}
    \newcommand{\SpecialCharTok}[1]{\textcolor[rgb]{0.25,0.44,0.63}{{#1}}}
    \newcommand{\VerbatimStringTok}[1]{\textcolor[rgb]{0.25,0.44,0.63}{{#1}}}
    \newcommand{\SpecialStringTok}[1]{\textcolor[rgb]{0.73,0.40,0.53}{{#1}}}
    \newcommand{\ImportTok}[1]{{#1}}
    \newcommand{\DocumentationTok}[1]{\textcolor[rgb]{0.73,0.13,0.13}{\textit{{#1}}}}
    \newcommand{\AnnotationTok}[1]{\textcolor[rgb]{0.38,0.63,0.69}{\textbf{\textit{{#1}}}}}
    \newcommand{\CommentVarTok}[1]{\textcolor[rgb]{0.38,0.63,0.69}{\textbf{\textit{{#1}}}}}
    \newcommand{\VariableTok}[1]{\textcolor[rgb]{0.10,0.09,0.49}{{#1}}}
    \newcommand{\ControlFlowTok}[1]{\textcolor[rgb]{0.00,0.44,0.13}{\textbf{{#1}}}}
    \newcommand{\OperatorTok}[1]{\textcolor[rgb]{0.40,0.40,0.40}{{#1}}}
    \newcommand{\BuiltInTok}[1]{{#1}}
    \newcommand{\ExtensionTok}[1]{{#1}}
    \newcommand{\PreprocessorTok}[1]{\textcolor[rgb]{0.74,0.48,0.00}{{#1}}}
    \newcommand{\AttributeTok}[1]{\textcolor[rgb]{0.49,0.56,0.16}{{#1}}}
    \newcommand{\InformationTok}[1]{\textcolor[rgb]{0.38,0.63,0.69}{\textbf{\textit{{#1}}}}}
    \newcommand{\WarningTok}[1]{\textcolor[rgb]{0.38,0.63,0.69}{\textbf{\textit{{#1}}}}}
    
    
    % Define a nice break command that doesn't care if a line doesn't already
    % exist.
    \def\br{\hspace*{\fill} \\* }
    % Math Jax compatability definitions
    \def\gt{>}
    \def\lt{<}
    % Document parameters
    \title{for\_niraj}
    
    
    

    % Pygments definitions
    
\makeatletter
\def\PY@reset{\let\PY@it=\relax \let\PY@bf=\relax%
    \let\PY@ul=\relax \let\PY@tc=\relax%
    \let\PY@bc=\relax \let\PY@ff=\relax}
\def\PY@tok#1{\csname PY@tok@#1\endcsname}
\def\PY@toks#1+{\ifx\relax#1\empty\else%
    \PY@tok{#1}\expandafter\PY@toks\fi}
\def\PY@do#1{\PY@bc{\PY@tc{\PY@ul{%
    \PY@it{\PY@bf{\PY@ff{#1}}}}}}}
\def\PY#1#2{\PY@reset\PY@toks#1+\relax+\PY@do{#2}}

\expandafter\def\csname PY@tok@nb\endcsname{\def\PY@tc##1{\textcolor[rgb]{0.00,0.50,0.00}{##1}}}
\expandafter\def\csname PY@tok@vg\endcsname{\def\PY@tc##1{\textcolor[rgb]{0.10,0.09,0.49}{##1}}}
\expandafter\def\csname PY@tok@ge\endcsname{\let\PY@it=\textit}
\expandafter\def\csname PY@tok@o\endcsname{\def\PY@tc##1{\textcolor[rgb]{0.40,0.40,0.40}{##1}}}
\expandafter\def\csname PY@tok@mo\endcsname{\def\PY@tc##1{\textcolor[rgb]{0.40,0.40,0.40}{##1}}}
\expandafter\def\csname PY@tok@kn\endcsname{\let\PY@bf=\textbf\def\PY@tc##1{\textcolor[rgb]{0.00,0.50,0.00}{##1}}}
\expandafter\def\csname PY@tok@nc\endcsname{\let\PY@bf=\textbf\def\PY@tc##1{\textcolor[rgb]{0.00,0.00,1.00}{##1}}}
\expandafter\def\csname PY@tok@gd\endcsname{\def\PY@tc##1{\textcolor[rgb]{0.63,0.00,0.00}{##1}}}
\expandafter\def\csname PY@tok@mf\endcsname{\def\PY@tc##1{\textcolor[rgb]{0.40,0.40,0.40}{##1}}}
\expandafter\def\csname PY@tok@c\endcsname{\let\PY@it=\textit\def\PY@tc##1{\textcolor[rgb]{0.25,0.50,0.50}{##1}}}
\expandafter\def\csname PY@tok@s1\endcsname{\def\PY@tc##1{\textcolor[rgb]{0.73,0.13,0.13}{##1}}}
\expandafter\def\csname PY@tok@nf\endcsname{\def\PY@tc##1{\textcolor[rgb]{0.00,0.00,1.00}{##1}}}
\expandafter\def\csname PY@tok@ne\endcsname{\let\PY@bf=\textbf\def\PY@tc##1{\textcolor[rgb]{0.82,0.25,0.23}{##1}}}
\expandafter\def\csname PY@tok@cp\endcsname{\def\PY@tc##1{\textcolor[rgb]{0.74,0.48,0.00}{##1}}}
\expandafter\def\csname PY@tok@kt\endcsname{\def\PY@tc##1{\textcolor[rgb]{0.69,0.00,0.25}{##1}}}
\expandafter\def\csname PY@tok@nd\endcsname{\def\PY@tc##1{\textcolor[rgb]{0.67,0.13,1.00}{##1}}}
\expandafter\def\csname PY@tok@kp\endcsname{\def\PY@tc##1{\textcolor[rgb]{0.00,0.50,0.00}{##1}}}
\expandafter\def\csname PY@tok@vi\endcsname{\def\PY@tc##1{\textcolor[rgb]{0.10,0.09,0.49}{##1}}}
\expandafter\def\csname PY@tok@cs\endcsname{\let\PY@it=\textit\def\PY@tc##1{\textcolor[rgb]{0.25,0.50,0.50}{##1}}}
\expandafter\def\csname PY@tok@sr\endcsname{\def\PY@tc##1{\textcolor[rgb]{0.73,0.40,0.53}{##1}}}
\expandafter\def\csname PY@tok@gi\endcsname{\def\PY@tc##1{\textcolor[rgb]{0.00,0.63,0.00}{##1}}}
\expandafter\def\csname PY@tok@se\endcsname{\let\PY@bf=\textbf\def\PY@tc##1{\textcolor[rgb]{0.73,0.40,0.13}{##1}}}
\expandafter\def\csname PY@tok@sc\endcsname{\def\PY@tc##1{\textcolor[rgb]{0.73,0.13,0.13}{##1}}}
\expandafter\def\csname PY@tok@m\endcsname{\def\PY@tc##1{\textcolor[rgb]{0.40,0.40,0.40}{##1}}}
\expandafter\def\csname PY@tok@nt\endcsname{\let\PY@bf=\textbf\def\PY@tc##1{\textcolor[rgb]{0.00,0.50,0.00}{##1}}}
\expandafter\def\csname PY@tok@w\endcsname{\def\PY@tc##1{\textcolor[rgb]{0.73,0.73,0.73}{##1}}}
\expandafter\def\csname PY@tok@mh\endcsname{\def\PY@tc##1{\textcolor[rgb]{0.40,0.40,0.40}{##1}}}
\expandafter\def\csname PY@tok@gr\endcsname{\def\PY@tc##1{\textcolor[rgb]{1.00,0.00,0.00}{##1}}}
\expandafter\def\csname PY@tok@si\endcsname{\let\PY@bf=\textbf\def\PY@tc##1{\textcolor[rgb]{0.73,0.40,0.53}{##1}}}
\expandafter\def\csname PY@tok@nn\endcsname{\let\PY@bf=\textbf\def\PY@tc##1{\textcolor[rgb]{0.00,0.00,1.00}{##1}}}
\expandafter\def\csname PY@tok@sx\endcsname{\def\PY@tc##1{\textcolor[rgb]{0.00,0.50,0.00}{##1}}}
\expandafter\def\csname PY@tok@err\endcsname{\def\PY@bc##1{\setlength{\fboxsep}{0pt}\fcolorbox[rgb]{1.00,0.00,0.00}{1,1,1}{\strut ##1}}}
\expandafter\def\csname PY@tok@k\endcsname{\let\PY@bf=\textbf\def\PY@tc##1{\textcolor[rgb]{0.00,0.50,0.00}{##1}}}
\expandafter\def\csname PY@tok@vc\endcsname{\def\PY@tc##1{\textcolor[rgb]{0.10,0.09,0.49}{##1}}}
\expandafter\def\csname PY@tok@sh\endcsname{\def\PY@tc##1{\textcolor[rgb]{0.73,0.13,0.13}{##1}}}
\expandafter\def\csname PY@tok@c1\endcsname{\let\PY@it=\textit\def\PY@tc##1{\textcolor[rgb]{0.25,0.50,0.50}{##1}}}
\expandafter\def\csname PY@tok@gu\endcsname{\let\PY@bf=\textbf\def\PY@tc##1{\textcolor[rgb]{0.50,0.00,0.50}{##1}}}
\expandafter\def\csname PY@tok@no\endcsname{\def\PY@tc##1{\textcolor[rgb]{0.53,0.00,0.00}{##1}}}
\expandafter\def\csname PY@tok@na\endcsname{\def\PY@tc##1{\textcolor[rgb]{0.49,0.56,0.16}{##1}}}
\expandafter\def\csname PY@tok@gs\endcsname{\let\PY@bf=\textbf}
\expandafter\def\csname PY@tok@kr\endcsname{\let\PY@bf=\textbf\def\PY@tc##1{\textcolor[rgb]{0.00,0.50,0.00}{##1}}}
\expandafter\def\csname PY@tok@gt\endcsname{\def\PY@tc##1{\textcolor[rgb]{0.00,0.27,0.87}{##1}}}
\expandafter\def\csname PY@tok@mi\endcsname{\def\PY@tc##1{\textcolor[rgb]{0.40,0.40,0.40}{##1}}}
\expandafter\def\csname PY@tok@go\endcsname{\def\PY@tc##1{\textcolor[rgb]{0.53,0.53,0.53}{##1}}}
\expandafter\def\csname PY@tok@bp\endcsname{\def\PY@tc##1{\textcolor[rgb]{0.00,0.50,0.00}{##1}}}
\expandafter\def\csname PY@tok@gp\endcsname{\let\PY@bf=\textbf\def\PY@tc##1{\textcolor[rgb]{0.00,0.00,0.50}{##1}}}
\expandafter\def\csname PY@tok@kd\endcsname{\let\PY@bf=\textbf\def\PY@tc##1{\textcolor[rgb]{0.00,0.50,0.00}{##1}}}
\expandafter\def\csname PY@tok@sd\endcsname{\let\PY@it=\textit\def\PY@tc##1{\textcolor[rgb]{0.73,0.13,0.13}{##1}}}
\expandafter\def\csname PY@tok@kc\endcsname{\let\PY@bf=\textbf\def\PY@tc##1{\textcolor[rgb]{0.00,0.50,0.00}{##1}}}
\expandafter\def\csname PY@tok@ss\endcsname{\def\PY@tc##1{\textcolor[rgb]{0.10,0.09,0.49}{##1}}}
\expandafter\def\csname PY@tok@gh\endcsname{\let\PY@bf=\textbf\def\PY@tc##1{\textcolor[rgb]{0.00,0.00,0.50}{##1}}}
\expandafter\def\csname PY@tok@sb\endcsname{\def\PY@tc##1{\textcolor[rgb]{0.73,0.13,0.13}{##1}}}
\expandafter\def\csname PY@tok@il\endcsname{\def\PY@tc##1{\textcolor[rgb]{0.40,0.40,0.40}{##1}}}
\expandafter\def\csname PY@tok@ni\endcsname{\let\PY@bf=\textbf\def\PY@tc##1{\textcolor[rgb]{0.60,0.60,0.60}{##1}}}
\expandafter\def\csname PY@tok@nv\endcsname{\def\PY@tc##1{\textcolor[rgb]{0.10,0.09,0.49}{##1}}}
\expandafter\def\csname PY@tok@s\endcsname{\def\PY@tc##1{\textcolor[rgb]{0.73,0.13,0.13}{##1}}}
\expandafter\def\csname PY@tok@nl\endcsname{\def\PY@tc##1{\textcolor[rgb]{0.63,0.63,0.00}{##1}}}
\expandafter\def\csname PY@tok@ow\endcsname{\let\PY@bf=\textbf\def\PY@tc##1{\textcolor[rgb]{0.67,0.13,1.00}{##1}}}
\expandafter\def\csname PY@tok@s2\endcsname{\def\PY@tc##1{\textcolor[rgb]{0.73,0.13,0.13}{##1}}}
\expandafter\def\csname PY@tok@cm\endcsname{\let\PY@it=\textit\def\PY@tc##1{\textcolor[rgb]{0.25,0.50,0.50}{##1}}}

\def\PYZbs{\char`\\}
\def\PYZus{\char`\_}
\def\PYZob{\char`\{}
\def\PYZcb{\char`\}}
\def\PYZca{\char`\^}
\def\PYZam{\char`\&}
\def\PYZlt{\char`\<}
\def\PYZgt{\char`\>}
\def\PYZsh{\char`\#}
\def\PYZpc{\char`\%}
\def\PYZdl{\char`\$}
\def\PYZhy{\char`\-}
\def\PYZsq{\char`\'}
\def\PYZdq{\char`\"}
\def\PYZti{\char`\~}
% for compatibility with earlier versions
\def\PYZat{@}
\def\PYZlb{[}
\def\PYZrb{]}
\makeatother


    % Exact colors from NB
    \definecolor{incolor}{rgb}{0.0, 0.0, 0.5}
    \definecolor{outcolor}{rgb}{0.545, 0.0, 0.0}



    
    % Prevent overflowing lines due to hard-to-break entities
    \sloppy 
    % Setup hyperref package
    \hypersetup{
      breaklinks=true,  % so long urls are correctly broken across lines
      colorlinks=true,
      urlcolor=urlcolor,
      linkcolor=linkcolor,
      citecolor=citecolor,
      }
    % Slightly bigger margins than the latex defaults
    
    \geometry{verbose,tmargin=1in,bmargin=1in,lmargin=1in,rmargin=1in}
    
    

    \begin{document}
    
    
    \maketitle
    
    

    
    \begin{Verbatim}[commandchars=\\\{\}]
{\color{incolor}In [{\color{incolor}121}]:} \PY{k+kn}{import} \PY{n+nn}{matplotlib}\PY{n+nn}{.}\PY{n+nn}{pyplot} \PY{k}{as} \PY{n+nn}{plt}
          \PY{k+kn}{import} \PY{n+nn}{numpy} \PY{k}{as} \PY{n+nn}{np}
          \PY{k+kn}{from} \PY{n+nn}{IPython}\PY{n+nn}{.}\PY{n+nn}{core}\PY{n+nn}{.}\PY{n+nn}{display} \PY{k}{import} \PY{n}{HTML}
          \PY{n}{HTML}\PY{p}{(}\PY{l+s}{\PYZdq{}\PYZdq{}\PYZdq{}}
          \PY{l+s}{\PYZlt{}style\PYZgt{}}
          
          \PY{l+s}{div.cell \PYZob{} /* Tunes the space between cells */}
          \PY{l+s}{margin\PYZhy{}top:1em;}
          \PY{l+s}{margin\PYZhy{}bottom:1em;}
          \PY{l+s}{\PYZcb{}}
          
          \PY{l+s}{div.text\PYZus{}cell\PYZus{}render h1 \PYZob{} /* Main titles bigger, centered */}
          \PY{l+s}{font\PYZhy{}size: 2.2em;}
          \PY{l+s}{line\PYZhy{}height:1.4em;}
          \PY{l+s}{text\PYZhy{}align:center;}
          \PY{l+s}{\PYZcb{}}
          
          \PY{l+s}{div.text\PYZus{}cell\PYZus{}render h2 \PYZob{} /*  Parts names nearer from text */}
          \PY{l+s}{margin\PYZhy{}bottom: \PYZhy{}0.4em;}
          \PY{l+s}{\PYZcb{}}
          
          
          \PY{l+s}{div.text\PYZus{}cell\PYZus{}render \PYZob{} /* Customize text cells */}
          \PY{l+s}{font\PYZhy{}family: }\PY{l+s}{\PYZsq{}}\PY{l+s}{Times New Roman}\PY{l+s}{\PYZsq{}}\PY{l+s}{;}
          \PY{l+s}{font\PYZhy{}size:1.5em;}
          \PY{l+s}{line\PYZhy{}height:1.4em;}
          \PY{l+s}{padding\PYZhy{}left:3em;}
          \PY{l+s}{padding\PYZhy{}right:3em;}
          \PY{l+s}{ }
          \PY{l+s}{\PYZcb{}}
          \PY{l+s}{\PYZlt{}/style\PYZgt{}}
          \PY{l+s}{\PYZdq{}\PYZdq{}\PYZdq{}}\PY{p}{)}
\end{Verbatim}


\begin{Verbatim}[commandchars=\\\{\}]
{\color{outcolor}Out[{\color{outcolor}121}]:} <IPython.core.display.HTML object>
\end{Verbatim}
            
    \section{What is Overlay?}\label{what-is-overlay}

    \subsubsection{Short Version}\label{short-version}

Overlay is a cryptocurrency that allows users to make and lose money on
nearly any streaming data. It recreates the dynamics of trading, but
without counterparties. Thus it completely solves the liquidity problems
which beset similar systems like cash-settled futures and scalar
prediction markets.

    \subsubsection{Longer Version}\label{longer-version}

There are some sources of streaming data (price data is a paradigm, but
it could be CPI numbers, weather data, crime rate, etc.) and holders of
OVL "buy" or "sell" these data points through some interface. The data
value at the time of the "trade" is saved to the blockchain. At a later
time when the "trade" is unwound the value of the streaming data is
queried again, compared to the value in the blockchain, and a simple
return is calculated. The original amount of OVL committed to the
"trade" is multiplied by the return. \emph{The OVL in a user's wallet is
minted or burned dynamically, based on the "trades" the user makes.}

    \section{What are Some Killer Use
Cases?}\label{what-are-some-killer-use-cases}

    Like an organism from a single cell, Overlay is based on a simple idea
which unfolds into a beautiful and complex system. Some consequences of
the system are:

\begin{itemize}
\tightlist
\item
  Overlay solves liquidity problems. Most data sources are not tradable,
  either on a futures market or a prediction market, because sellers and
  buyers must be matched. Significant trader interest must already exist
  to make such markets tradable, and so very few of such possible
  markets are actual.
\item
  The Overlay system can easily set fees negative, thus paying users to
  trade and providing a powerful incentive to attract users in the early
  days and drive network effects. These negative fees can also be
  offered whenever the system is deflationary and can afford to attract
  more traders.
\item
  Global data sources and interest in monetizing data will continue to
  grow, with no end in sight. Overlay has enormous potential to serve as
  a go-to platform which offers investors, hedgers, and speculators
  exposure to financial instruments they cannot get anywhere else.
\item
  As a consequence of the previous point, a mature Overlay market would
  establish OVL as an entirely new class of financial derivative, in its
  own right. It would also allow higher-level derivatives like futures
  and options to be defined on extremely illiquid markets.
\item
  Overlay allows for the easy synthetic construction of a crypto (or
  any) portfolio. It thus simulates a universal cryptocurrency wallet
  without the need to maintain multiple nodes or keep money on an
  exchange.
\item
  Because there are no counterparties, there is no price impact. Any OVL
  trade, of any amount, settles at a single price.
\end{itemize}

    \section{From Problem to Solution}\label{from-problem-to-solution}

    To provide the greatest context for the idea, and to help understand how
it is innovative, we will start with a well-known problem, and using
alchemy transform it into a series of increasingly tractable problems.

    \subsubsection{\texorpdfstring{PROBLEM: How can we monetize streaming
data that is just a
number?}{PROBLEM:  How can we monetize streaming data that is just a number?}}\label{problem-how-can-we-monetize-streaming-data-that-is-just-a-number}

Price is streaming data formed by buyers and sellers, and is extremely
interesting because it is monetized. Most streaming data is not formed
in this way and is not monetizable. Obvious examples abound, the most
relevant ones are data values derived from prices, like the S\&P 500 and
VIX.

    \subsubsection{\texorpdfstring{SOLUTION:Cash-settled futures markets and
scalar prediction
markets.}{SOLUTION: Cash-settled futures markets and scalar prediction markets.}}\label{solution-cash-settled-futures-markets-and-scalar-prediction-markets.}

Most futures markets have physical settlement: a coffee future contract
can result in delivery of a truck of 200 lb burlap sacks. Cash-settled
futures result in payouts. When the contract expires, all parties settle
in cash at the data level, and the result is to allow for synthetic
ownership, i.e. monetization of the underyling data stream. Scalar
prediction markets are essentially cash-settled futures: they set a band
within which a data value is predicted to remain, and upon expiry all
parties settle in cash. This solution works, but it gives rise to a new
problem:

    \subsubsection{\texorpdfstring{PROBLEM:Liquidity is only available for
hugely popular data
streams.}{PROBLEM: Liquidity is only available for hugely popular data streams.}}\label{problem-liquidity-is-only-available-for-hugely-popular-data-streams.}

Only a handful of the possible data streams can be monetized with
futures or prediction markets because of the liquidity problem caused by
the nature of exchange itself: every buyer needs to be matched with a
seller. To my knowledge there have been no serious attempts to address
this problem.

    \subsubsection{\texorpdfstring{SOLUTION:A cryptocurrency which allows
synthetic ownership without
exchange.}{SOLUTION: A cryptocurrency which allows synthetic ownership without exchange.}}\label{solution-a-cryptocurrency-which-allows-synthetic-ownership-without-exchange.}

The novelty of Overlay is that it allows for any data stream to be
monetized and it solves the liquidity problem by doing away with
exchange. The cryptocurrency plays the role of a trading platform like
Interactive Brokers, which gives users access to various tradable
instruments. It is not quite right to say there are no counterparties,
because now the counterparty is the entire user base, the Overlay system
itself. Overlay solves the first and second problems completely, but it
replaces them with another hard problem.

    \subsubsection{\texorpdfstring{PROBLEM How to control the currency
supply of OVL to avoid
hyperinflation?}{PROBLEM  How to control the currency supply of OVL to avoid hyperinflation?}}\label{problem-how-to-control-the-currency-supply-of-ovl-to-avoid-hyperinflation}

Overlay is only interesting if users can redeem it for their favorite
currency without too much slippage. Thus there needs to be liquid OVL
secondary markets on exchanges. However, if new OVL supply floods the
system then the price of OVL on exchanges can plummet, causing the users
of OVL to abandon the system.

    \subsubsection{\texorpdfstring{SOLUTION:Caps.}{SOLUTION: Caps.}}\label{solution-caps.}

If the amount of new OVL available to mint at a given time is capped,
there is no inflation problem. The max\_supply of OVL is greater than or
equal to the current\_supply of OVL, and the difference is the buffer,
the total amount users can pull from the system. The max\_supply should
be dynamic. For example it could grow linearly (10 OVL are added per
block), nonlinearly in an error-function-like curve (BTC and ETH
issuance), or be constant until the buffer is distressed (quantitative
easing). The max\_supply could even shrink in some situations. Users of
OVL can know that the currency supply will always be less than or equal
to a given number.

    \#\#\# PROBLEM:How to assure that the buffer stays large enough?

We have actually now come full circle, because this is a liquidity
problem, but of quite a different sort than the one in problem \#2. If
the buffer drops to zero the system must react. If the buffer stays at
zero no users can redeem their positions, and it becomes unlikely then
that they will enter new positions. If users even suspect that this is
about to happen, the system is likely to experience a crisis.

    \subsubsection{\texorpdfstring{SOLUTION:Trader caps and dynamic
fees.}{SOLUTION: Trader caps and dynamic fees.}}\label{solution-trader-caps-and-dynamic-fees.}

The best situation is that the buffer never gets small and either stays
roughly constant or is always growing. The first problem to solve is the
case of black swan windfalls when a trader makes millions of OVL by
betting 1 OVL. (This can happen if a trader shorts any amount at any
price and the data feed crashes to near zero.) This problem is easy
though: all traders have a max\_bet, and a max\_payout. The max\_bet can
depend on the number of active traders and the current buffer size,
among other things. The max\_payout can be large, like 10*max\_bet. This
will help assure that the buffer is not destroyed by black swan
windfalls. Even if we assume moderate wins per trade, however, it is by
no means obvious that the buffer will still stay large enough. It is
quite surprising that dynamic fees solve this problem.

    \section{Modeling Dynamic Fees}\label{modeling-dynamic-fees}

    \subsection{Data}\label{data}

    We start by setting up a data feed which we can control. The histogram
below shows the distribution, which is just random normal. The positive
values are green and the negative are red.

    \begin{Verbatim}[commandchars=\\\{\}]
{\color{incolor}In [{\color{incolor}122}]:} \PY{n}{distribution} \PY{o}{=} \PY{n}{np}\PY{o}{.}\PY{n}{random}\PY{o}{.}\PY{n}{normal}
          \PY{n}{samples} \PY{o}{=} \PY{p}{[}\PY{p}{]}
          \PY{k}{for} \PY{n}{i} \PY{o+ow}{in} \PY{n+nb}{range}\PY{p}{(}\PY{l+m+mi}{10000}\PY{p}{)}\PY{p}{:}
              \PY{n}{samples}\PY{o}{.}\PY{n}{append}\PY{p}{(}\PY{n}{distribution}\PY{p}{(}
                  \PY{n}{loc}\PY{o}{=}\PY{l+m+mi}{0}\PY{p}{,}     \PY{c}{\PYZsh{}the average value of the distribution}
                  \PY{n}{scale}\PY{o}{=}\PY{l+m+mi}{1}\PY{p}{)}   \PY{c}{\PYZsh{}the standard deviation (spread, or width) of the distribution }
              \PY{p}{)}
          \PY{n}{pos} \PY{o}{=} \PY{p}{[}\PY{n}{x} \PY{k}{for} \PY{n}{x} \PY{o+ow}{in} \PY{n}{samples} \PY{k}{if} \PY{n}{x}\PY{o}{\PYZgt{}}\PY{l+m+mi}{0}\PY{p}{]}
          \PY{n}{neg} \PY{o}{=} \PY{p}{[}\PY{n}{x} \PY{k}{for} \PY{n}{x} \PY{o+ow}{in} \PY{n}{samples} \PY{k}{if} \PY{n}{x}\PY{o}{\PYZlt{}}\PY{o}{=}\PY{l+m+mi}{0}\PY{p}{]}
          \PY{n}{hist} \PY{o}{=} \PY{n}{plt}\PY{o}{.}\PY{n}{hist}\PY{p}{(}\PY{n}{pos}\PY{p}{,} \PY{n}{bins}\PY{o}{=}\PY{l+m+mi}{25}\PY{p}{,} \PY{n}{color}\PY{o}{=}\PY{l+s}{\PYZsq{}}\PY{l+s}{g}\PY{l+s}{\PYZsq{}}\PY{p}{,} \PY{n}{alpha}\PY{o}{=}\PY{o}{.}\PY{l+m+mi}{45}\PY{p}{)}
          \PY{n}{hist} \PY{o}{=} \PY{n}{plt}\PY{o}{.}\PY{n}{hist}\PY{p}{(}\PY{n}{neg}\PY{p}{,} \PY{n}{bins}\PY{o}{=}\PY{l+m+mi}{25}\PY{p}{,} \PY{n}{color}\PY{o}{=}\PY{l+s}{\PYZsq{}}\PY{l+s}{r}\PY{l+s}{\PYZsq{}}\PY{p}{,} \PY{n}{alpha}\PY{o}{=}\PY{o}{.}\PY{l+m+mi}{5}\PY{p}{)}
          \PY{n}{plt}\PY{o}{.}\PY{n}{axvline}\PY{p}{(}\PY{l+m+mi}{0}\PY{p}{,} \PY{n}{c}\PY{o}{=}\PY{l+s}{\PYZsq{}}\PY{l+s}{r}\PY{l+s}{\PYZsq{}}\PY{p}{,} \PY{n}{label}\PY{o}{=}\PY{l+s}{\PYZsq{}}\PY{l+s}{mean}\PY{l+s}{\PYZsq{}}\PY{p}{)}
          \PY{n}{plt}\PY{o}{.}\PY{n}{axvline}\PY{p}{(}\PY{l+m+mi}{1}\PY{p}{,} \PY{n}{c}\PY{o}{=}\PY{l+s}{\PYZsq{}}\PY{l+s}{k}\PY{l+s}{\PYZsq{}}\PY{p}{,} \PY{n}{label}\PY{o}{=}\PY{l+s}{\PYZsq{}}\PY{l+s}{1 std}\PY{l+s}{\PYZsq{}}\PY{p}{)}
          \PY{n}{plt}\PY{o}{.}\PY{n}{axvline}\PY{p}{(}\PY{o}{\PYZhy{}}\PY{l+m+mi}{1}\PY{p}{,} \PY{n}{c}\PY{o}{=}\PY{l+s}{\PYZsq{}}\PY{l+s}{k}\PY{l+s}{\PYZsq{}}\PY{p}{)}
          \PY{n}{plt}\PY{o}{.}\PY{n}{legend}\PY{p}{(}\PY{p}{)}
          \PY{n}{plt}\PY{o}{.}\PY{n}{show}\PY{p}{(}\PY{p}{)}
\end{Verbatim}


    \begin{center}
    \adjustimage{max size={0.9\linewidth}{0.9\paperheight}}{for_niraj_files/for_niraj_19_0.png}
    \end{center}
    { \hspace*{\fill} \\}
    
    We get our price values through time by taking the cumulative sum of
these deltas. Disregard the negative prices, that's just a fluke of the
randomness we can control for by starting the market at a different
price.

    \begin{Verbatim}[commandchars=\\\{\}]
{\color{incolor}In [{\color{incolor}73}]:} \PY{n}{cumsum} \PY{o}{=} \PY{n}{np}\PY{o}{.}\PY{n}{cumsum}\PY{p}{(}\PY{n}{samples}\PY{p}{)}
         \PY{n}{plt}\PY{o}{.}\PY{n}{plot}\PY{p}{(}\PY{n}{cumsum}\PY{p}{)}
         \PY{n}{plt}\PY{o}{.}\PY{n}{show}\PY{p}{(}\PY{p}{)}
\end{Verbatim}


    \begin{center}
    \adjustimage{max size={0.9\linewidth}{0.9\paperheight}}{for_niraj_files/for_niraj_21_0.png}
    \end{center}
    { \hspace*{\fill} \\}
    
    We can change the volatility by changing the standard distribution,
causing much larger price swings

    \begin{Verbatim}[commandchars=\\\{\}]
{\color{incolor}In [{\color{incolor}76}]:} \PY{n}{distribution} \PY{o}{=} \PY{n}{np}\PY{o}{.}\PY{n}{random}\PY{o}{.}\PY{n}{normal}
         \PY{n}{samples2} \PY{o}{=} \PY{p}{[}\PY{p}{]}
         \PY{k}{for} \PY{n}{i} \PY{o+ow}{in} \PY{n+nb}{range}\PY{p}{(}\PY{l+m+mi}{10000}\PY{p}{)}\PY{p}{:}
             \PY{n}{samples2}\PY{o}{.}\PY{n}{append}\PY{p}{(}\PY{n}{distribution}\PY{p}{(}
                 \PY{n}{loc}\PY{o}{=}\PY{l+m+mi}{0}\PY{p}{,}     \PY{c}{\PYZsh{}the average value of the distribution}
                 \PY{n}{scale}\PY{o}{=}\PY{l+m+mi}{10}\PY{p}{)}   \PY{c}{\PYZsh{}the standard deviation (spread, or width) of the distribution }
             \PY{p}{)}
         \PY{n}{pos} \PY{o}{=} \PY{p}{[}\PY{n}{x} \PY{k}{for} \PY{n}{x} \PY{o+ow}{in} \PY{n}{samples2} \PY{k}{if} \PY{n}{x}\PY{o}{\PYZgt{}}\PY{l+m+mi}{0}\PY{p}{]}
         \PY{n}{neg} \PY{o}{=} \PY{p}{[}\PY{n}{x} \PY{k}{for} \PY{n}{x} \PY{o+ow}{in} \PY{n}{samples2} \PY{k}{if} \PY{n}{x}\PY{o}{\PYZlt{}}\PY{o}{=}\PY{l+m+mi}{0}\PY{p}{]}
         \PY{n}{hist} \PY{o}{=} \PY{n}{plt}\PY{o}{.}\PY{n}{hist}\PY{p}{(}\PY{n}{pos}\PY{p}{,} \PY{n}{bins}\PY{o}{=}\PY{l+m+mi}{25}\PY{p}{,} \PY{n}{color}\PY{o}{=}\PY{l+s}{\PYZsq{}}\PY{l+s}{g}\PY{l+s}{\PYZsq{}}\PY{p}{,} \PY{n}{alpha}\PY{o}{=}\PY{o}{.}\PY{l+m+mi}{45}\PY{p}{)}
         \PY{n}{hist} \PY{o}{=} \PY{n}{plt}\PY{o}{.}\PY{n}{hist}\PY{p}{(}\PY{n}{neg}\PY{p}{,} \PY{n}{bins}\PY{o}{=}\PY{l+m+mi}{25}\PY{p}{,} \PY{n}{color}\PY{o}{=}\PY{l+s}{\PYZsq{}}\PY{l+s}{r}\PY{l+s}{\PYZsq{}}\PY{p}{,} \PY{n}{alpha}\PY{o}{=}\PY{o}{.}\PY{l+m+mi}{5}\PY{p}{)}
         \PY{n}{plt}\PY{o}{.}\PY{n}{axvline}\PY{p}{(}\PY{l+m+mi}{0}\PY{p}{,} \PY{n}{c}\PY{o}{=}\PY{l+s}{\PYZsq{}}\PY{l+s}{r}\PY{l+s}{\PYZsq{}}\PY{p}{,} \PY{n}{label}\PY{o}{=}\PY{l+s}{\PYZsq{}}\PY{l+s}{mean}\PY{l+s}{\PYZsq{}}\PY{p}{)}
         \PY{n}{plt}\PY{o}{.}\PY{n}{axvline}\PY{p}{(}\PY{l+m+mi}{10}\PY{p}{,} \PY{n}{c}\PY{o}{=}\PY{l+s}{\PYZsq{}}\PY{l+s}{k}\PY{l+s}{\PYZsq{}}\PY{p}{,} \PY{n}{label}\PY{o}{=}\PY{l+s}{\PYZsq{}}\PY{l+s}{1 std}\PY{l+s}{\PYZsq{}}\PY{p}{)}
         \PY{n}{plt}\PY{o}{.}\PY{n}{axvline}\PY{p}{(}\PY{o}{\PYZhy{}}\PY{l+m+mi}{10}\PY{p}{,} \PY{n}{c}\PY{o}{=}\PY{l+s}{\PYZsq{}}\PY{l+s}{k}\PY{l+s}{\PYZsq{}}\PY{p}{)}
         \PY{n}{plt}\PY{o}{.}\PY{n}{legend}\PY{p}{(}\PY{p}{)}
         \PY{n}{plt}\PY{o}{.}\PY{n}{show}\PY{p}{(}\PY{p}{)}
\end{Verbatim}


    \begin{center}
    \adjustimage{max size={0.9\linewidth}{0.9\paperheight}}{for_niraj_files/for_niraj_23_0.png}
    \end{center}
    { \hspace*{\fill} \\}
    
    The previous chart is also plotted for comparison.

    \begin{Verbatim}[commandchars=\\\{\}]
{\color{incolor}In [{\color{incolor}81}]:} \PY{n}{cumsum} \PY{o}{=} \PY{n}{np}\PY{o}{.}\PY{n}{cumsum}\PY{p}{(}\PY{n}{samples}\PY{p}{)}
         \PY{n}{cumsum2} \PY{o}{=} \PY{n}{np}\PY{o}{.}\PY{n}{cumsum}\PY{p}{(}\PY{n}{samples2}\PY{p}{)}
         \PY{n}{plt}\PY{o}{.}\PY{n}{plot}\PY{p}{(}\PY{n}{cumsum2}\PY{p}{,} \PY{n}{label}\PY{o}{=}\PY{l+s}{\PYZsq{}}\PY{l+s}{std:10}\PY{l+s}{\PYZsq{}}\PY{p}{)}
         \PY{n}{plt}\PY{o}{.}\PY{n}{plot}\PY{p}{(}\PY{n}{cumsum}\PY{p}{,} \PY{n}{label}\PY{o}{=}\PY{l+s}{\PYZsq{}}\PY{l+s}{std:1}\PY{l+s}{\PYZsq{}}\PY{p}{)}
         \PY{n}{plt}\PY{o}{.}\PY{n}{legend}\PY{p}{(}\PY{p}{)}
         \PY{n}{plt}\PY{o}{.}\PY{n}{show}\PY{p}{(}\PY{p}{)}
\end{Verbatim}


    \begin{center}
    \adjustimage{max size={0.9\linewidth}{0.9\paperheight}}{for_niraj_files/for_niraj_25_0.png}
    \end{center}
    { \hspace*{\fill} \\}
    
    We can make the market trend up or down by changing the mean value of
the distribution. A shift of a single standard deviation upwards has a
profound impact on the market through time.

    \begin{Verbatim}[commandchars=\\\{\}]
{\color{incolor}In [{\color{incolor}55}]:} \PY{n}{samples} \PY{o}{=} \PY{p}{[}\PY{p}{]}
         \PY{k}{for} \PY{n}{i} \PY{o+ow}{in} \PY{n+nb}{range}\PY{p}{(}\PY{l+m+mi}{10000}\PY{p}{)}\PY{p}{:}
             \PY{n}{samples}\PY{o}{.}\PY{n}{append}\PY{p}{(}\PY{n}{distribution}\PY{p}{(}
                 \PY{n}{loc}\PY{o}{=}\PY{l+m+mi}{1}\PY{p}{,}     \PY{c}{\PYZsh{}the average value of the distribution}
                 \PY{n}{scale}\PY{o}{=}\PY{l+m+mi}{1}\PY{p}{)}   \PY{c}{\PYZsh{}the standard deviation (spread, or width) of the distribution }
             \PY{p}{)}
         \PY{n}{pos} \PY{o}{=} \PY{p}{[}\PY{n}{x} \PY{k}{for} \PY{n}{x} \PY{o+ow}{in} \PY{n}{samples} \PY{k}{if} \PY{n}{x}\PY{o}{\PYZgt{}}\PY{l+m+mi}{0}\PY{p}{]}
         \PY{n}{neg} \PY{o}{=} \PY{p}{[}\PY{n}{x} \PY{k}{for} \PY{n}{x} \PY{o+ow}{in} \PY{n}{samples} \PY{k}{if} \PY{n}{x}\PY{o}{\PYZlt{}}\PY{o}{=}\PY{l+m+mi}{0}\PY{p}{]}
         \PY{n}{hist} \PY{o}{=} \PY{n}{plt}\PY{o}{.}\PY{n}{hist}\PY{p}{(}\PY{n}{pos}\PY{p}{,} \PY{n}{bins}\PY{o}{=}\PY{l+m+mi}{25}\PY{p}{,} \PY{n}{color}\PY{o}{=}\PY{l+s}{\PYZsq{}}\PY{l+s}{g}\PY{l+s}{\PYZsq{}}\PY{p}{,} \PY{n}{alpha}\PY{o}{=}\PY{o}{.}\PY{l+m+mi}{45}\PY{p}{)}
         \PY{n}{hist} \PY{o}{=} \PY{n}{plt}\PY{o}{.}\PY{n}{hist}\PY{p}{(}\PY{n}{neg}\PY{p}{,} \PY{n}{bins}\PY{o}{=}\PY{l+m+mi}{12}\PY{p}{,} \PY{n}{color}\PY{o}{=}\PY{l+s}{\PYZsq{}}\PY{l+s}{r}\PY{l+s}{\PYZsq{}}\PY{p}{,} \PY{n}{alpha}\PY{o}{=}\PY{o}{.}\PY{l+m+mi}{5}\PY{p}{)}
         \PY{n}{plt}\PY{o}{.}\PY{n}{axvline}\PY{p}{(}\PY{l+m+mi}{1}\PY{p}{,} \PY{n}{c}\PY{o}{=}\PY{l+s}{\PYZsq{}}\PY{l+s}{r}\PY{l+s}{\PYZsq{}}\PY{p}{,} \PY{n}{label}\PY{o}{=}\PY{l+s}{\PYZsq{}}\PY{l+s}{mean}\PY{l+s}{\PYZsq{}}\PY{p}{)}
         \PY{n}{plt}\PY{o}{.}\PY{n}{axvline}\PY{p}{(}\PY{l+m+mi}{2}\PY{p}{,} \PY{n}{c}\PY{o}{=}\PY{l+s}{\PYZsq{}}\PY{l+s}{k}\PY{l+s}{\PYZsq{}}\PY{p}{,} \PY{n}{label}\PY{o}{=}\PY{l+s}{\PYZsq{}}\PY{l+s}{1 std}\PY{l+s}{\PYZsq{}}\PY{p}{)}
         \PY{n}{plt}\PY{o}{.}\PY{n}{axvline}\PY{p}{(}\PY{l+m+mi}{0}\PY{p}{,} \PY{n}{c}\PY{o}{=}\PY{l+s}{\PYZsq{}}\PY{l+s}{k}\PY{l+s}{\PYZsq{}}\PY{p}{)}
         \PY{n}{plt}\PY{o}{.}\PY{n}{legend}\PY{p}{(}\PY{p}{)}
         \PY{n}{plt}\PY{o}{.}\PY{n}{show}\PY{p}{(}\PY{p}{)}
\end{Verbatim}


    \begin{center}
    \adjustimage{max size={0.9\linewidth}{0.9\paperheight}}{for_niraj_files/for_niraj_27_0.png}
    \end{center}
    { \hspace*{\fill} \\}
    
    Not random globally, looks just like the line \(y = x\).

    \begin{Verbatim}[commandchars=\\\{\}]
{\color{incolor}In [{\color{incolor}37}]:} \PY{n}{cumsum} \PY{o}{=} \PY{n}{np}\PY{o}{.}\PY{n}{cumsum}\PY{p}{(}\PY{n}{samples}\PY{p}{)}
         \PY{n}{plt}\PY{o}{.}\PY{n}{plot}\PY{p}{(}\PY{n}{cumsum}\PY{p}{)}
         \PY{n}{plt}\PY{o}{.}\PY{n}{show}\PY{p}{(}\PY{p}{)}
\end{Verbatim}


    \begin{center}
    \adjustimage{max size={0.9\linewidth}{0.9\paperheight}}{for_niraj_files/for_niraj_29_0.png}
    \end{center}
    { \hspace*{\fill} \\}
    
    Locally, the randomness is more evident.

    \begin{Verbatim}[commandchars=\\\{\}]
{\color{incolor}In [{\color{incolor}67}]:} \PY{n}{cumsum} \PY{o}{=} \PY{n}{np}\PY{o}{.}\PY{n}{cumsum}\PY{p}{(}\PY{n}{samples}\PY{p}{[}\PY{l+m+mi}{1110}\PY{p}{:}\PY{l+m+mi}{1125}\PY{p}{]}\PY{p}{)}
         \PY{n}{plt}\PY{o}{.}\PY{n}{plot}\PY{p}{(}\PY{n}{cumsum}\PY{p}{)}
         \PY{n}{plt}\PY{o}{.}\PY{n}{show}\PY{p}{(}\PY{p}{)}
\end{Verbatim}


    \begin{center}
    \adjustimage{max size={0.9\linewidth}{0.9\paperheight}}{for_niraj_files/for_niraj_31_0.png}
    \end{center}
    { \hspace*{\fill} \\}
    
    Even a very slight shift in the mean has a profound effect. This is a
counterintuitive feature of statistics and why statistical reasoning is
simultaneously so powerful and so poorly understood by most people.
Below we shift the mean up by only .05

    \begin{Verbatim}[commandchars=\\\{\}]
{\color{incolor}In [{\color{incolor}103}]:} \PY{n}{samples} \PY{o}{=} \PY{p}{[}\PY{p}{]}
          \PY{k}{for} \PY{n}{i} \PY{o+ow}{in} \PY{n+nb}{range}\PY{p}{(}\PY{l+m+mi}{10000}\PY{p}{)}\PY{p}{:}
              \PY{n}{samples}\PY{o}{.}\PY{n}{append}\PY{p}{(}\PY{n}{distribution}\PY{p}{(}
                  \PY{n}{loc}\PY{o}{=}\PY{o}{.}\PY{l+m+mi}{05}\PY{p}{,}     \PY{c}{\PYZsh{}the average value of the distribution}
                  \PY{n}{scale}\PY{o}{=}\PY{l+m+mi}{1}\PY{p}{)}     \PY{c}{\PYZsh{}the standard deviation (spread, or width) of the distribution }
              \PY{p}{)}
          \PY{n}{pos} \PY{o}{=} \PY{p}{[}\PY{n}{x} \PY{k}{for} \PY{n}{x} \PY{o+ow}{in} \PY{n}{samples} \PY{k}{if} \PY{n}{x}\PY{o}{\PYZgt{}}\PY{l+m+mi}{0}\PY{p}{]}
          \PY{n}{neg} \PY{o}{=} \PY{p}{[}\PY{n}{x} \PY{k}{for} \PY{n}{x} \PY{o+ow}{in} \PY{n}{samples} \PY{k}{if} \PY{n}{x}\PY{o}{\PYZlt{}}\PY{o}{=}\PY{l+m+mi}{0}\PY{p}{]}
          \PY{n}{hist} \PY{o}{=} \PY{n}{plt}\PY{o}{.}\PY{n}{hist}\PY{p}{(}\PY{n}{pos}\PY{p}{,} \PY{n}{bins}\PY{o}{=}\PY{l+m+mi}{25}\PY{p}{,} \PY{n}{color}\PY{o}{=}\PY{l+s}{\PYZsq{}}\PY{l+s}{g}\PY{l+s}{\PYZsq{}}\PY{p}{,} \PY{n}{alpha}\PY{o}{=}\PY{o}{.}\PY{l+m+mi}{45}\PY{p}{)}
          \PY{n}{hist} \PY{o}{=} \PY{n}{plt}\PY{o}{.}\PY{n}{hist}\PY{p}{(}\PY{n}{neg}\PY{p}{,} \PY{n}{bins}\PY{o}{=}\PY{l+m+mi}{25}\PY{p}{,} \PY{n}{color}\PY{o}{=}\PY{l+s}{\PYZsq{}}\PY{l+s}{r}\PY{l+s}{\PYZsq{}}\PY{p}{,} \PY{n}{alpha}\PY{o}{=}\PY{o}{.}\PY{l+m+mi}{5}\PY{p}{)}
          \PY{n}{plt}\PY{o}{.}\PY{n}{axvline}\PY{p}{(}\PY{o}{.}\PY{l+m+mi}{05}\PY{p}{,} \PY{n}{c}\PY{o}{=}\PY{l+s}{\PYZsq{}}\PY{l+s}{r}\PY{l+s}{\PYZsq{}}\PY{p}{,} \PY{n}{label}\PY{o}{=}\PY{l+s}{\PYZsq{}}\PY{l+s}{mean}\PY{l+s}{\PYZsq{}}\PY{p}{)}
          \PY{n}{plt}\PY{o}{.}\PY{n}{axvline}\PY{p}{(}\PY{l+m+mf}{1.1}\PY{p}{,} \PY{n}{c}\PY{o}{=}\PY{l+s}{\PYZsq{}}\PY{l+s}{k}\PY{l+s}{\PYZsq{}}\PY{p}{,} \PY{n}{label}\PY{o}{=}\PY{l+s}{\PYZsq{}}\PY{l+s}{1 std}\PY{l+s}{\PYZsq{}}\PY{p}{)}
          \PY{n}{plt}\PY{o}{.}\PY{n}{axvline}\PY{p}{(}\PY{o}{\PYZhy{}}\PY{o}{.}\PY{l+m+mi}{9}\PY{p}{,} \PY{n}{c}\PY{o}{=}\PY{l+s}{\PYZsq{}}\PY{l+s}{k}\PY{l+s}{\PYZsq{}}\PY{p}{)}
          \PY{n}{plt}\PY{o}{.}\PY{n}{legend}\PY{p}{(}\PY{p}{)}
          \PY{n}{plt}\PY{o}{.}\PY{n}{show}\PY{p}{(}\PY{p}{)}
\end{Verbatim}


    \begin{center}
    \adjustimage{max size={0.9\linewidth}{0.9\paperheight}}{for_niraj_files/for_niraj_33_0.png}
    \end{center}
    { \hspace*{\fill} \\}
    
    This is still far from random gobally, the slope is approximately 5/100
or .05

    \begin{Verbatim}[commandchars=\\\{\}]
{\color{incolor}In [{\color{incolor}104}]:} \PY{n}{cumsum} \PY{o}{=} \PY{n}{np}\PY{o}{.}\PY{n}{cumsum}\PY{p}{(}\PY{n}{samples}\PY{p}{)}
          \PY{n}{plt}\PY{o}{.}\PY{n}{plot}\PY{p}{(}\PY{n}{cumsum}\PY{p}{)}
          \PY{n}{plt}\PY{o}{.}\PY{n}{show}\PY{p}{(}\PY{p}{)}
\end{Verbatim}


    \begin{center}
    \adjustimage{max size={0.9\linewidth}{0.9\paperheight}}{for_niraj_files/for_niraj_35_0.png}
    \end{center}
    { \hspace*{\fill} \\}
    
    Locally the randomness is much more evident

    \begin{Verbatim}[commandchars=\\\{\}]
{\color{incolor}In [{\color{incolor}105}]:} \PY{n}{cumsum} \PY{o}{=} \PY{n}{np}\PY{o}{.}\PY{n}{cumsum}\PY{p}{(}\PY{n}{samples}\PY{p}{[}\PY{l+m+mi}{1110}\PY{p}{:}\PY{l+m+mi}{1125}\PY{p}{]}\PY{p}{)}
          \PY{n}{plt}\PY{o}{.}\PY{n}{plot}\PY{p}{(}\PY{n}{cumsum}\PY{p}{)}
          \PY{n}{plt}\PY{o}{.}\PY{n}{show}\PY{p}{(}\PY{p}{)}
\end{Verbatim}


    \begin{center}
    \adjustimage{max size={0.9\linewidth}{0.9\paperheight}}{for_niraj_files/for_niraj_37_0.png}
    \end{center}
    { \hspace*{\fill} \\}
    
    \subsection{Traders}\label{traders}

    We now add traders to these markets. The logic here can get arbitrarily
complex. Below we have a trader who, at each time step, has a \(33\%\)
chance of trading. Upon \emph{entry} a side is chosen by coin toss, and
at each trade the trader commits all capital.

The \texttt{price\_action} plot shows the price along with the trades
made. The \texttt{P\&L} chart shows the real-time, marked-to-market
profit and loss of the trader.

We run the model for one year, supposing that each step is a day.

    

    It may seem unrealistic to pursue this model, but it is useful because
trader luck and skill (and the opposite) can be simulated without
writing any trading strategies. This is done by increasing the
probability of trades going in the direction of a controlled trend. In
the below chart, the market is trending up and the trader is twice as
likely to go long as short upon entry.

    

    We may also control the frequency of trades through the entry and exit
probabilities. Keeping everything form the previous chart except
reducing the percentage of a trade to \(2\%\) we get the following.
(Note that the \texttt{LongEnter} strategy has a 1/3 chance of going
short, so despite having a winning strategy this trader gets unlucky.)

    

    \subsection{Agent-Based Modeling}\label{agent-based-modeling}

    What we are actually interested in are statistical features of this
model, so we add many more traders, and we do many runs of the same
parameters to look for regularities.

Below, the top chart is the earned wealth of the 100 traders \emph{on
aggregate} (i.e. the sum of all traders' wealth). The middle chart is
the buffer, and the bottom chart is the currency supply. The charts on
the right are rotated histograms of the outcomes.

Each line on each chart is a separate run of the model. Each run is a
complete reset and is independent of all other runs. As is clear, these
three quantities are essentially the same thing: changes in earned
wealth just are changes in currency supply, and changes in the buffer
are the same, but inverted. We bother charting them all because there
are ways of decoupling them.

The results are entirely expected: the outcomes roughly follow a random
normal distribution, with the mean outcome being no different than the
initial state.

    

    The \texttt{issue} variable describes how many OVL are added per time
step to the \texttt{max\_supply}. Such 'issuance' does not actually
affect the OVL currency supply, but it does affect the buffer. Changing
the \texttt{issue} parameter to 10 and otherwise leaving the model
unchanged, results in a decoupling of the \texttt{buffer} from the
\texttt{earned\_wealth} and \texttt{curency\_supply}.

    

    Now, if we set \texttt{issue} to 0 again, but change the
\texttt{free\_fee} to 10 basis points (\(.1\%\) of each trade size), all
variables are coupled again, but the result is profound:

    

    Over the course of a year, 10 basis points shifts the average
performance down \(10\%\). This is for traders who trade, on balance,
once every three days. If we reduce the trading frequency to once per
month, we obtain a decidedly more modest result (eyeballing it, it looks
like fees take about \(1\%\) of \texttt{earned\_wealth} on average):

    

    If we run the exact same model for two years, the result of the fees is
more apparent (fees take about \(2\%\) of \texttt{earned\_wealth} now):

    

    Finally, let us reset the trading frequency to once every three days,
and run the model for a year. If we combine fees and issue, we get the
expected result: while \texttt{earned\_wealth} decreases by roughly
\(10\%\), the \texttt{buffer} increases by roughly \(15\%\).

    

    All of the above examples used a random market and random entry and exit
parameters. How do fees affect the situation when traders, on aggregate,
are winning? Below we look at a \texttt{LongEntry} strategy (i.e. twice
as likely to go long than short), with a rising market, in the absence
of fees and issue.

    

    If we keep everything the same, and include a fee of 10 basis points,
the pressue on the system is alleviated.

    

    If we reduce the trading frequency by a factor of ten (one trade per
month on average), the fees are less effective.

    

    And if we have traders use a \texttt{AlwaysLongEnter} strategy in a
market which trends strongly upwards, along with a low trading
probability, we get a simulation of incredible trader skill, and a
system which experiences a great deal of pressure:

    

    Doubling the fees, we get

    

    Tripling the fees (which is now getting a little expensive):

    

    The higher fees go, the fewer users we will have and the fewer new
trades will be entered. If, instead of raising fees to deal with trader
luck and skill, we allow for some inflation in the system, we can
maintain the buffer. The following chart shows what happens when we
charge 25 basis points fee, and issue 50 OVL per day to the
\texttt{max\_supply}. The result is a system which experiences, at
worst, \(4\%\) inflation per year, and a trader population that makes
\(40\%\) return over the year. The buffer, in the worst case, loses only
\(20\%\).

    

    Running the model for two years gives the same result: the buffer is
sustained in the face of what will probably be moderate inflation.

    

    \section{Conclusion}\label{conclusion}

    We conclude that fees alone are sufficient to sustain and protect the
system when traders are unskilled or unlucky. In the case that traders
collectively get very lucky, or have unusual skill, a combination of
fees and inflation of the currency allows the system to maintain the
buffer. Such periods are stressful, and there should be other safeguards
available to deal with market shocks. We believe that dynamic fees based
on a careful (and continually optimized) model for the optimal fee, will
be a necessary component of the Overlay system. The encouraging analysis
above shows a way for future work on the monetary policy.


    % Add a bibliography block to the postdoc
    
    
    
    \end{document}
